\documentclass[10pt,a4paper,titlepage,oneside]{article}
\usepackage{LabProtocol}

\exercise{Exercise II}

% enter your data here
\authors{
	Vorname Nachname, Matr. Nr. 0123456 \par
	{\small e0123456@student.tuwien.ac.at} \par
}


\begin{document}

\maketitle


%████████╗ █████╗ ███████╗██╗  ██╗     ██╗
%╚══██╔══╝██╔══██╗██╔════╝██║ ██╔╝    ███║
%   ██║   ███████║███████╗█████╔╝     ╚██║
%   ██║   ██╔══██║╚════██║██╔═██╗      ██║
%   ██║   ██║  ██║███████║██║  ██╗     ██║
%   ╚═╝   ╚═╝  ╚═╝╚══════╝╚═╝  ╚═╝     ╚═╝
\Task{VGA Graphics Controller}

\begin{qa}{VGA Oscilloscope Measurements}

	\begin{figure}[h!]
		\centering
		% \includegraphics[width=1.0\linewidth]{your filename here}
		\dummyimage
		\caption{Line measurement with cursors marking the length (duration) of the whole line}
	\end{figure}
	
	\begin{figure}[h!]
		\centering
		% \includegraphics[width=1.0\linewidth]{your filename here}
		\dummyimage
		\caption{Line measurement with cursors marking the length (duration) of the horizontal synchronization pulse}
	\end{figure}
	

\end{qa}
%%%%%%%%%%%%%%%%%%%%%%%%%%%%%%%%%%%%%%%%%%%%%%%%%%%%%%%%%%%%%%%%%%%%%%%%%%%%%%%%


%████████╗ █████╗ ███████╗██╗  ██╗    ██████╗ 
%╚══██╔══╝██╔══██╗██╔════╝██║ ██╔╝    ╚════██╗
%   ██║   ███████║███████╗█████╔╝      █████╔╝
%   ██║   ██╔══██║╚════██║██╔═██╗     ██╔═══╝ 
%   ██║   ██║  ██║███████║██║  ██╗    ███████╗
%   ╚═╝   ╚═╝  ╚═╝╚══════╝╚═╝  ╚═╝    ╚══════╝
\Task{Tetris Game}

\begin{qa}{Briefly describe the architecture of your \textsf{tetris\_game} module. Are there any submodules? What is their purpose? How many FSMs did you use?}
add your explanation here (approximately 8-10 sentences, you can also include figures) 
\end{qa}

\subsection{submodules for drawing}
The submodules used for the drawing of blocks and strings on the screen are located in tetrisdrawerspkg.vhd
The smallest unit of the implementation is a TBBBLOCK. It defines which bitblit to use when drawing it, and whether it is empty or not. 
A TBBROW is made of multiple TBBBLOCKS.
A TBBMATRIX is made of multiple TBBROWS. 
The blockdrawer is used to draw a specific bitblit to a specified position on the screen. It is not used directly in the tetrisarchitecture.
The blocklinedrawer is used to draw a line of blocks across the screen, provided a startingpoint, a tbbblock to use, a length, and a direction(horizontal or vertical).
Inside the architecture, it used to draw the borders of the map.
The blockmatrixdrawer is used to draw a tbbblockmatrix, starting at the specified x and y coordinates.
The stringdrawer is used to print the labels for the score, removed-lines and next-tetromino.   

\subsection{submodules for game logic}
The submodules used for the game logic are located in the tetrisgamelogicpkg.vhd
The addtetrominohandler takes a blockmatrix, a tetromino, the coordinates of the tetromino inside the matrix, and the rotation of the tetromino as input.
It adds all solid blocks of the tetromino to the matrix, and returns the new matrix. 
The rowsfullhandler is used for checking for full rows, removing them, and shifting the remaining rows accordingly. It also provides information about the number of rows that were removed.
The timer is a simple counter that counts the number of clockcycles since it was last started, and sets its ouput to '1' after the specified amount of time has passed.
It is both used for the automatic downwards movement of the tetromino, aswell as for measuring the time between different pitches for the synthesizers. 

\subsection{submodules for music}
The methods and constants used for the music are located in the tetrisaudiopkg.vhd
Here, a basic 2-channel arrangement of the tetris-theme is provided, aswell as some functions matching the different pitches to 8-bit vectors for the high- and lowtime of the synthesizers.
Shouldn't you have tried implementing Task 1 before wasting time with adding music to the game? Yes, yes I should have.

\subsection{submodules architecture}
As for the tetrisarchitecture itself, its heart is the state.matrix object.
It stores information about which blocks are solid on the map, aswell as which type of bitblit should be used when drawing them. The tetromino-collider acceses the matrix and 
the current tetromino.
Couldn't you just have used the on-chip-memory to store the matrix? Yes, yes I could have. In hinsight, this would have been the better choice, as all submodules which have a matrix as output and input require 
a huge amount of data-connections and flip-flops, wheras storing the matrix inside the on-chip memory would have been the much more efficient option.
When the collider registers a collision, the addtetrominohandler is started, which adds the current tetromino to the blockmatrix.
After that, the rowsfullhandler is started, which checks for full rows and removes them accordingly. 


%%%%%%%%%%%%%%%%%%%%%%%%%%%%%%%%%%%%%%%%%%%%%%%%%%%%%%%%%%%%%%%%%%%%%%%%%%%%%%%%

%████████╗ █████╗ ███████╗██╗  ██╗    ██████╗ 
%╚══██╔══╝██╔══██╗██╔════╝██║ ██╔╝    ╚════██╗
%   ██║   ███████║███████╗█████╔╝      █████╔╝
%   ██║   ██╔══██║╚════██║██╔═██╗      ╚═══██╗
%   ██║   ██║  ██║███████║██║  ██╗    ██████╔╝
%   ╚═╝   ╚═╝  ╚═╝╚══════╝╚═╝  ╚═╝    ╚═════╝ 
\Task{Bonus: SignalTap Measurement}

\begin{qa}{Trigger Condition}
	\begin{figure}[h!]
		\centering
		% \includegraphics[width=1.0\linewidth]{your filename here}
		\dummyimage
		\caption{Screenshot showing the trigger condition}
	\end{figure}
\end{qa}
%%%%%%%%%%%%%%%%%%%%%%%%%%%%%%%%%%%%%%%%%%%%%%%%%%%%%%%%%%%%%%%%%%%%%%%%%%%%%%%%

\begin{qa}{Measurement Screenshot}
	\begin{figure}[h!]
		\centering
		% \includegraphics[width=1.0\linewidth]{your filename here}
		\dummyimage
		\caption{Screenshot showing at least the first 4 instructions (and their associated data items) issued to the graphics controller during one frame by the \textsf{tetris\_game} module.}
	\end{figure}
\end{qa}
%%%%%%%%%%%%%%%%%%%%%%%%%%%%%%%%%%%%%%%%%%%%%%%%%%%%%%%%%%%%%%%%%%%%%%%%%%%%%%%%

\begin{qa}{Instruction Decoding}
	\begin{center}
	\scriptsize
	\begin{tabular}{p{0.05\linewidth}p{0.2\linewidth}p{0.25\linewidth}p{0.40\linewidth}}
		Command & Operands                                 & Instruction Name & Description \\\hline\hline
		0x..    & \valuelist{0x0001}                       & ??               & ...\\\hline
		0x..    & \valuelist{0x0001,0x0002}                & ??               & ...\\\hline
		0x..    & \valuelist{0x0001,0x0002,0x0003}         & ??               & ...\\\hline
		0x..    & \valuelist{0x0001,0x0002,0x0003,0x0004}  & ??               & ...\\\hline
	\end{tabular}
	\end{center}
\end{qa}
%%%%%%%%%%%%%%%%%%%%%%%%%%%%%%%%%%%%%%%%%%%%%%%%%%%%%%%%%%%%%%%%%%%%%%%%%%%%%%%%

\end{document}
