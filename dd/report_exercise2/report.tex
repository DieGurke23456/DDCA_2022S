\documentclass[10pt,a4paper,titlepage,oneside]{article}
\usepackage{LabProtocol}

\exercise{Exercise II}

% enter your data here
\authors{
	Vorname Nachname, Matr. Nr. 0123456 \par
	{\small e0123456@student.tuwien.ac.at} \par
}


\begin{document}

\maketitle


%████████╗ █████╗ ███████╗██╗  ██╗     ██╗
%╚══██╔══╝██╔══██╗██╔════╝██║ ██╔╝    ███║
%   ██║   ███████║███████╗█████╔╝     ╚██║
%   ██║   ██╔══██║╚════██║██╔═██╗      ██║
%   ██║   ██║  ██║███████║██║  ██╗     ██║
%   ╚═╝   ╚═╝  ╚═╝╚══════╝╚═╝  ╚═╝     ╚═╝
\Task{VGA Graphics Controller}

\begin{qa}{VGA Oscilloscope Measurements}

	\begin{figure}[h!]
		\centering
		% \includegraphics[width=1.0\linewidth]{your filename here}
		\dummyimage
		\caption{Line measurement with cursors marking the length (duration) of the whole line}
	\end{figure}
	
	\begin{figure}[h!]
		\centering
		% \includegraphics[width=1.0\linewidth]{your filename here}
		\dummyimage
		\caption{Line measurement with cursors marking the length (duration) of the horizontal synchronization pulse}
	\end{figure}
	

\end{qa}
%%%%%%%%%%%%%%%%%%%%%%%%%%%%%%%%%%%%%%%%%%%%%%%%%%%%%%%%%%%%%%%%%%%%%%%%%%%%%%%%


%████████╗ █████╗ ███████╗██╗  ██╗    ██████╗ 
%╚══██╔══╝██╔══██╗██╔════╝██║ ██╔╝    ╚════██╗
%   ██║   ███████║███████╗█████╔╝      █████╔╝
%   ██║   ██╔══██║╚════██║██╔═██╗     ██╔═══╝ 
%   ██║   ██║  ██║███████║██║  ██╗    ███████╗
%   ╚═╝   ╚═╝  ╚═╝╚══════╝╚═╝  ╚═╝    ╚══════╝
\Task{Tetris Game}

\begin{qa}{Briefly describe the architecture of your \textsf{tetris\_game} module. Are there any submodules? What is their purpose? How many FSMs did you use?}
add your explanation here (approximately 8-10 sentences, you can also include figures) ... 
\end{qa}
%%%%%%%%%%%%%%%%%%%%%%%%%%%%%%%%%%%%%%%%%%%%%%%%%%%%%%%%%%%%%%%%%%%%%%%%%%%%%%%%

%████████╗ █████╗ ███████╗██╗  ██╗    ██████╗ 
%╚══██╔══╝██╔══██╗██╔════╝██║ ██╔╝    ╚════██╗
%   ██║   ███████║███████╗█████╔╝      █████╔╝
%   ██║   ██╔══██║╚════██║██╔═██╗      ╚═══██╗
%   ██║   ██║  ██║███████║██║  ██╗    ██████╔╝
%   ╚═╝   ╚═╝  ╚═╝╚══════╝╚═╝  ╚═╝    ╚═════╝ 
\Task{Bonus: SignalTap Measurement}

\begin{qa}{Trigger Condition}
	\begin{figure}[h!]
		\centering
		% \includegraphics[width=1.0\linewidth]{your filename here}
		\dummyimage
		\caption{Screenshot showing the trigger condition}
	\end{figure}
\end{qa}
%%%%%%%%%%%%%%%%%%%%%%%%%%%%%%%%%%%%%%%%%%%%%%%%%%%%%%%%%%%%%%%%%%%%%%%%%%%%%%%%

\begin{qa}{Measurement Screenshot}
	\begin{figure}[h!]
		\centering
		% \includegraphics[width=1.0\linewidth]{your filename here}
		\dummyimage
		\caption{Screenshot showing at least the first 4 instructions (and their associated data items) issued to the graphics controller during one frame by the \textsf{tetris\_game} module.}
	\end{figure}
\end{qa}
%%%%%%%%%%%%%%%%%%%%%%%%%%%%%%%%%%%%%%%%%%%%%%%%%%%%%%%%%%%%%%%%%%%%%%%%%%%%%%%%

\begin{qa}{Instruction Decoding}
	\begin{center}
	\scriptsize
	\begin{tabular}{p{0.05\linewidth}p{0.2\linewidth}p{0.25\linewidth}p{0.40\linewidth}}
		Command & Operands                                 & Instruction Name & Description \\\hline\hline
		0x..    & \valuelist{0x0001}                       & ??               & ...\\\hline
		0x..    & \valuelist{0x0001,0x0002}                & ??               & ...\\\hline
		0x..    & \valuelist{0x0001,0x0002,0x0003}         & ??               & ...\\\hline
		0x..    & \valuelist{0x0001,0x0002,0x0003,0x0004}  & ??               & ...\\\hline
	\end{tabular}
	\end{center}
\end{qa}
%%%%%%%%%%%%%%%%%%%%%%%%%%%%%%%%%%%%%%%%%%%%%%%%%%%%%%%%%%%%%%%%%%%%%%%%%%%%%%%%

\end{document}
